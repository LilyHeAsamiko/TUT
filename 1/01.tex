\documentclass[a4paper,12pt]{article}
\usepackage[latin1]{inputenc}              %  [utf8] is also acceptable
\usepackage[T1]{fontenc}                   %  if you need scandinavian letters

\title{ Weekly Exercise 1: ``Undefined'' in Verification -- A New Point of View}
\author{Shubo Yan }
\date{30.8.2017}

\begin{document}
\maketitle
\begin{abstract}
Partially defined processes have been used in verification more than ten years. When the notion ``undefined'' is added in to process \emph{definition} formalism, it must also be added to the used semantics to ensure that advanced verification methods, like the structural LTS construction, still work correct. Until now this has been done separately to each semantics but the resulting theories have became complicated.

In this paper we discuss about the notion ``undefined'' in the level of strong bisimilarity. We prove a theorem that express the correctness of slightly modified LTS construction. Modified, since it uses the notion ``undefined''. The method expects little about the used abstract semantics and therefore suits to all commonly used semantics. We prove that the assumption it makes is the weakest possible that still guarantees the correctness of the method. We also present some other variations of the method. One of them can be used even without tool support for the notion ``undefined''.


\end{abstract}

\maketitle

\tableofcontents

\section{Introduction}
The first section is usually Introduction. It can be written in several ways. However, usually the introduction contains the followings:
(1) what is the research topic the work is part of, 
(2) some crucial problems of that topic, 
(3) some published solutions to these problems, 
(4) your research problem, (5) your solution method, 
(6) the most important results, and 
(7) presentation of the structure of the article section by section.

\section{Labelled State Transition System}

In the second section the background theory the work is based on is presented. It is often named as Theory, Background, or Background Theory, but nothing prevents you using your imagination to give it a name that describes it content.

\subsection{Basic definitions}

\subsection{Strong bisimilarity between LTSCs}

\subsection{Two cut states related precongruences}
In the next sections you develop your own theory, compare existing theories, algorithms, etc. as you tell in the introduction that you do in your solution. 
\section{Verification with Cut States}
In the last sections of the method sections you can also introduce possible measurement methods and devices, and justify the used measurement data. The measurements can be presented in a table or a graph, and the analysis is made based on them.


\subsection{Adding cut states to the model}

\subsection{Cut states in compositional LTS construction}

\subsection{Marking the boundaries of the model}

\section{Conclusions}
The last section is usually Conclusions (or Summary) where you can repeat what is done, collect the results of the analysis, and make the conclusions.
 
\section*{Acknowledgements}
It is custom to bring up by name the persons and organsations that have helped you or have funded the work.

\section*{Bibliography}
\addcontentsline{toc}{section}{Bibliography}
Normally here should be References but this time we have a list of literature related to the work.


\appendix
\section{Strong Bisimilarity}

In appendices you can tell things you have no room in the article but still want to show for the reader.

\section{Proofs}
In the appendices you usually put measurements and the hardest proofs in more detailed than in the actual article.

\end{document}
