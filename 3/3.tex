\documentclass[a4paper,12pt]{article}
\usepackage[latin1]{inputenc}
\usepackage[T1]{fontenc}

\setlength\overfullrule{5pt}

\newcommand{\pois}[1]{}
\title{ Weekly Exercise 3}
\author{Shubo Yan }
\date{12.9.2017}
\begin{document}
	\maketitle
\newpage
\section{Question 1}
\subsection*{\textsc{Arthur C. Clarke}}

{\LARGE \textsc {Sir Arthur Charles Clarke}\par}

\begin{itemize}	
\item Lived at 16 December 1917--19 March 2008
\item Alma mater: King's College London
\item Career:
\begin{enumerate}
\item a British science  action writer,
\item science writer and futurist,
\item inventor,
\item undersea explorer,
\end{enumerate}	
and
\begin{enumerate}
\item[5.] television series host
\end{enumerate}	
\end{itemize}	

He is famous for being co-writer of the screenplay for the 1968 film 2001: A Space Odyssey, widely considered to be one of the most influential films of all time.
\begin{quote}
\emph{Clarke was a science writer, who was both an avid populariser of space travel and a futurist of uncanny ability. On these subjects he wrote over a dozen books and many essays, which appeared in various popular magazines.}
\end{quote}
In 1961 he was awarded the \textbf{Kalinga Prize}, an award which is given by UNESCO for popularizing science. These along with his science fiction writings eventually earned him the moniker ``Prophet of the Space Age''. His other science fiction writings earned him a number of Hugo and Nebula awards, which along with a large readership made him one of the towering figures of science fiction.

For many years \underline{Clarke, Robert Heinlein and Isaac Asimov} were known as the ``Big Three'' of science fiction.
\pagebreak

\section{Question 2}

\subsection*{\centering Biography}
\begin{center}
Early years
\end{center}

Clarke was born in Minehead, Somerset, England, and grew up in nearby Bishops Lydeard.

\begin{flushright}
As a boy, he grew up on a farm enjoying stargazing, fossil collecting, and reading American science  ction pulp magazines.
\end{flushright}

\begin{flushleft}
He received his secondary education at Huish Grammar school in Taunton. Early in uences included dinosaur cigarette cards, which led to an enthusiasm for fossils starting about 1925.
\end{flushleft}

\begin{itemize}
\item \raggedright Clarke attributed his interest in science fiction to reading three items: the November 1928 issue of Amazing Stories, in 1929; Last and First Men by Olaf Stapledon in 1930; and The Conquest of Space by David Lasser in 1931.\par
\end{itemize}

{\raggedright In his teens, he joined the Junior Astronomical Association and contributed \par}
{\raggedleft to Urania, the society's journal, which was edited in Glasgow by Marion Eadie.\par}

\noindent This paragraph has the normal -- or default -- justification, but notice that it is not indented.
\pagebreak

\section{Question 3}

{\large 
When is Arthur C. Clarke born?\hrulefill\\
What is the name of the university he graduated?\hrulefill\\
List the most important books he has read (according to him):\\
before 1931 (3) \hrulefill\hrulefill\hrulefill\ and at 1931(1)\hrulefill}

\section{Question 4}
\vspace{\stretch{1}}
This {\large large size text is the last line of the page!\par}\pagebreak
\newpage
\section{Question 5}
\newpage
%\vspace*{\fill}
%\hspace{\stretch{1}}This line is in the middle of the page!\hspace{\stretch{1}}
%\vspace*{\fill}
%\pagebreak

\vspace*{\stretch{1}}
\hspace{\stretch{1}}This line is in the middle of the page!\hspace{\stretch{1}}
\vspace*{\stretch{1}}
\pagebreak

\section{Question 6}

\begin{tabbing}
	
Program\qquad \=: \= \TeX\\[5pt]
Author(s) \>: \> Donald Knuth\\[5pt]
Manuals \>:\\
\quad\= The Advanced \TeX Book \quad\= David Salomon
\quad\=Springer-Verlag\kill\\
\>\textsf{Name}\>\textsf{Author}\>\textsf{Publisher}\\[8pt]
\>The\TeX Book \>Donald Knuth \>Addison-Wesley\\[5pt]
\>The Advanced \TeX \ Book \>David Salomon\>Springer-Verlag\\[5pt]

\end{tabbing}
\section{Question 7}

{\small
\begin{tabbing}	
\hspace*{15mm}\=\hspace*{10mm}\=
\hspace*{5mm}\=\hspace*{5mm}\=\hspace*{5mm}\=\kill	
\textbf{public int}[ ] InsertionSort(int[ ] data)\{\\
\>\textbf{int} len = data.length;\\
\>\textbf{int} key = 0;\\
\>\textbf{int} i = 0;\\
\>\textbf{for}(\textbf{int} j = 1; j<len;j++)\{\\
\>\>key = data[j];\\
\>\>i = j-1;\\
\>\>\textbf{while}(i>=0 \&\& data[i]>key)\{\\
\>\>\>\>data[i+1] = data[i];\\
\>\>\>\>i = i-1;\\
\>\>\>\>data[i+1]=key;\\
\>\>\textbf{\}}\\
\>\}\\
\>\textbf{return} data;\\
\}
\end{tabbing}	
}

\end{document}
