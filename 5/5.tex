\documentclass[a4paper,12pt]{article}
\usepackage[latin1]{inputenc}
\usepackage[T1]{fontenc}
\usepackage{amssymb}
\usepackage{amsmath}
\usepackage[all,cmtip]{xy}

\setlength\overfullrule{5pt}

\newcommand{\pois}[1]{}
\title{ Weekly Exercise 5}
\author{Shubo Yan }
\date{27.9.2017}
\begin{document}
\maketitle
\newpage
\section{Question 1}
\begin{tabular}{l}
$\sin A\cos B = \left( \sin\left(A-B \right) + \sin\left(A+B\right)\right) / 2$
\\ \\
\fbox{$\displaystyle\sin A \cos B = \left(\sin\left(A-B \right)+\sin\left( A+B\right)\right)/2$}
\\ \\
$\sin A\cos B= \frac{\sin\left(A-B\right)+\sin\left(A+B\right)}{2}$
\\ \\
\fbox{$\displaystyle\sin A\cos B= \frac{\sin\left(A-B\right)+\sin\left(A+B\right)}{2}$}
\\ \\
$\sin A\cos B= \frac{1}{2}\left(\sin\left(A-B\right)+\sin\left(A+B\right)\right)$
\\ \\
\fbox{$\displaystyle\sin A\cos B= \frac{1}{2}\left(\sin\left(A-B\right)+\sin\left(A+B\right)\right)$}
\\ \\
$\sin A\cos B = \frac{1}{2}\bigl(\sin\left(A-B\right)+\sin\left(A+B\right)\bigr)$
\\ \\
\fbox{$\displaystyle\sin A\cos B = \frac{1}{2}\bigl(\sin\left(A-B\right)+\sin\left(A+B\right)\bigr)$}
\\ \\
$e^\phi=\cos\phi+i\sin\phi$ \qquad   
\fbox{$e\displaystyle^\phi=\cos\phi+i\sin\phi$}
\\ \\
$f(x)=\frac{d}{dx}\left(\int_0^x f(u)du\right)$ \qquad
\fbox{$\displaystyle f(x)=\frac{d}{dx}\left(\int_0^x f(u)du\right)$}
\\ \\
$\frac{d}{dx}\arctan \bigl(\sin\left(x^2\right)\bigr)=-2\frac{\cos\left(x^2\right)x}{-2+\bigl(\cos\left(x^2\right)\bigr)^2}$
\\ \\
\fbox{$\displaystyle\frac{d}{dx}\arctan \bigl(\sin\left(x^2\right)\bigr)=-2\frac{\cos\left(x^2\right)x}{-2+\bigl(\cos\left(x^2\right)\bigr)^2}$}
\\ \\
$\int_0^\infty e^{-x^2}=\frac{\sqrt{\pi}}{2}$ \qquad 
\fbox{$\displaystyle\int_{0}^\infty e^{-x^{2}}=\frac{\sqrt{\pi}}{2}$}
\end{tabular}
\pagebreak

\section{Question 2}

Let $A$ be a finite set and  $n\in\mathbb{N}$.
\begin{itemize}
\item $A^2 \stackrel{\textrm{def}}{=} A\times A$ is the \emph{cartesian product} of $A$ by itself.
\item If $A=\{a,b\}$ then $A^2 = \{(a,a),(a,b),(b,a),(b,b)\}.$
\item $A^n \stackrel{\textrm{def}}{=} \overbrace{ A\times A\times \cdots\times A}^\textrm{$n$ times}$ is $A$ to the power of $n$ or the set of all $n$-tuples that can be produced from $A$. Especially $A^0=\{()\}$.
\item The \emph{powerset} of $A$ is $\mathcal{P}(A)= \Bigl\{B\Bigm|B\subseteq A\Bigl\}$, the set of all subsets of $A$. It is often denoted also as $2^A$.
\item When the elements of $A$ are interpreted as characters, it is a custom to leave the parenthesis and commas.
It is also a custom to denote $() =\varepsilon$.
\item $A^*\stackrel{\textrm{def}}{=}A^0\cup A^1\cup\cdots \displaystyle=\lim_{n \to \infty}\bigcup_{i=0}^n A^i$. The elements of $A^*$:n are all \emph{finite} (strings). 
But, if $|A|\ne 0$ then $|A^*|=\infty$.	
\item $A^\omega= \underbrace{A\times A\times\cdots}_\text{$\infty$ times}$ is the set of all $\infty$-tuples of $A$ (if $A= \emptyset$ then $A^\omega=\emptyset$).
\item $A^\infty\stackrel{\textrm{def}}{=}A^*\cup A^\omega$, the set of all finite and infinite tuples that can be produced from $A$.
\item The \emph{catenation} of strings $\alpha,\beta$ can be defined so that 
$\alpha\beta$ is the catena\-tion of $\alpha$ and $\beta$ if and only if  $\{\alpha\beta\}=\{\alpha\}\times\{\beta\}$.
\end{itemize}
\pagebreak

\section{Question 3}

\[
\bullet\hspace{-0.75em}\bigcirc \Longleftrightarrow\circ 
\overset{\longrightarrow}
{\underset{\longleftarrow} 
{\sideset{_\nwarrow^\nearrow}
{_\swarrow^\searrow}\bigodot}}\bullet\Longrightarrow 
\circ\hspace{-0.75em}\bigcirc
\]


\section{Question 4}

\[
  \frac{1+\sqrt{5}}{2}=
   \cfrac{1}{1+
	\cfrac{1}{1+
     \cfrac{1}{1+\ddots
}}}
\]

\[
  \pi=3+
   \cfrac{1}{7+
	\cfrac{1}{15+
	 \cfrac{1}{1+
	  \cfrac{1}{1+\ddots
}}}}
\]

\[
  \sqrt{x} =1 + \frac{x-1}{1+\sqrt{x}}=1+
   \cfrac{x-1}{2+
	\cfrac{x-1}{2+
	 \cfrac{x-1}{2+
	  \cfrac{x-1}{2+\ddots
}}}}
\]

\section{Question 5}

\begin{quote}
  \begin{tabbing}
	$\mathcal{C}(e,\Theta,\mathbf{r})$\quad\=\kill
	  $D(a,\mathbf{r})$\>$=\bigl\{ z\in \mathbb{C}\bigm| |z-a| 
	  < \mathbf{r}\big\}$,\\
	  $\mathsf{seg} (a,\textbf{r})$\>$=\bigl\{z\in\mathbb{C}\bigm|$ \=$\Im
	  z=\Im a \land |z-a|<\mathbf{r}\bigr\}$,\\
	  $c(e,\Theta,\mathbf{r})$\>$=\bigl\{(x,y)\in \mathbb{C}\bigm|$ \=$|x-e|
	  <\tan\Theta\:\land$\\
	  \>\> $0<y<\mathbf{r}\bigr\}$,and\\
	  $\mathcal{C}(E,\Theta,\mathbf{r})$\>$=\displaystyle\bigcup_{e\in E} c(e, \Theta,\mathbf{r})$.
	\end{tabbing}
\end{quote}
\end{document}